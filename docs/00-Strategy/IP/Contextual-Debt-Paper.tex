\documentclass{article}
\usepackage{graphicx}
\usepackage{amsmath}
\usepackage{hyperref}
\usepackage[utf8]{inputenc}

\title{The Unknowable Code: A Framework for Measuring and Mitigating Software Liability}
\author{AgentBeats Team}
\date{January 2026}

\begin{document}

\maketitle

\begin{abstract}
A new and insidious form of liability is silently accumulating within the digital infrastructure of the global economy. We have spent 20 years managing \textbf{Technical Debt}---a failure of the "how." But the next decade of \textit{liability} comes from \textbf{Contextual Debt}---a "failure of the 'why'". This is a massive, unmanaged risk, creating "amnesiac systems" that are ticking time bombs in every enterprise. The "context-free, AI-generated code" and fragmented microservices that power modern business are systematically eroding the human \textit{intent} and \textit{rationale} behind them.

\textbf{AgentBeats} is our response to this crisis. It is the world's first \textbf{Contextual Integrity Benchmark}, a system designed to measure the "Contextual Debt" of AI-generated code before it enters production.

This report formalizes the hypothesis that the greatest source of liability in the next decade of software will be the unmanageable accumulation of \textbf{Contextual Debt}.
\end{abstract}

\section{Introduction}

\textbf{Contextual Debt} is the compounding liability a software organization incurs from a lack of discernible human intent, architectural rationale, and domain-specific knowledge within its codebase.

This is not a failure of \textit{implementation} (the "how"), like traditional technical debt, but a more profound failure of \textit{intent} (the "why"). It manifests as a quantifiable drag on developer velocity and a direct threat to system security, creating subtle logic flaws that are invisible to traditional analysis tools. While accelerated by the proliferation of context-free, AI-generated code, its accumulation is a systemic risk that can only be managed through proactive disciplines that codify intent, automated governance architectures that preserve it, and new quantitative benchmarks like the \textbf{Contextual Integrity Score (CIS)}.

This analysis establishes a critical distinction between traditional technical debt and this emerging liability. Technical debt is a failure of the "how"---a suboptimal implementation chosen as a deliberate, short-term tradeoff for speed \cite{1}. Contextual Debt, in stark contrast, is a failure of the "why"---the evaporation of the purpose, rationale, and business logic that underpins the code \cite{3}. While the former makes a system difficult to change, the latter makes it dangerous to touch.

The \textbf{AgentBeats} platform implements the protocols described in this paper: the \textbf{Contextual Integrity Score (CIS)}, the \textbf{Agent-as-a-Judge} architecture, and the \textbf{Decision Bill of Materials (DBOM)}. It provides the first empirical method for quantifying the "why" of software, turning abstract liability into a measurable, manageable metric.

\section{Defining Contextual Debt: The Liability of Amnesiac Systems}

Contextual Debt represents a fundamental shift in how we must conceptualize liability in software. It moves beyond the mechanics of implementation to the very essence of a system's purpose. Where technical debt describes a system that is poorly constructed, Contextual Debt describes a system that has forgotten why it was built. It is the liability of unknowable code.

\subsection{The Anatomy of Contextual Debt}

At its core, Contextual Debt is "the accumulated, compounding liability from all the un-captured, siloed, or difficult-to-find information across the software development lifecycle" \cite{3}. It is the organizational friction and risk generated by amnesiac systems. Every time a team member is forced to halt productive work to manually reconstruct the history and purpose of a piece of code, they are paying an interest tax on this debt \cite{3}. This liability is built upon three pillars of missing context:

\begin{enumerate}
    \item \textbf{Lack of Discernible Human Intent:} This is the loss of the original purpose, business goals, and user needs that drove the creation of a feature or system. A developer might encounter a complex algorithm for calculating customer discounts but find no record of the business rules it is meant to enforce. Why was this specific formula chosen? Was it to comply with a regional regulation, to support a temporary marketing campaign, or to handle a specific type of enterprise client? Without this intent, any modification to the algorithm risks violating a critical business assumption, potentially leading to revenue loss or customer dissatisfaction \cite{21}.
    \item \textbf{Lack of Architectural Rationale:} This refers to the undocumented and unrecoverable reasons behind significant design choices. Modern systems are a series of complex tradeoffs: the choice of a particular database technology (e.g., SQL vs. NoSQL), the design of an API contract, the decision to use a message queue instead of direct calls, or the selection of a specific cloud service \cite{22}. When the rationale for these decisions is lost, the architecture becomes a collection of sacred, unchangeable artifacts. Teams become afraid to refactor or replace components because they do not know what constraints or assumptions the original choice was meant to satisfy. The system becomes brittle and impossible to evolve safely.
    \item \textbf{Lack of Domain-Specific Knowledge:} This is the evaporation of the nuanced understanding of the business domain from the codebase itself. Software is a model of a real-world process, and that model is only as good as its understanding of the domain. For example, in an e-commerce system, the distinction between a "Customer" (someone who has made a purchase) and a "Prospect" (someone who has not) is a critical piece of domain knowledge \cite{23}. If this distinction is not explicitly encoded in the system's types and variable names, but instead exists only as an implicit assumption in a few scattered functions, the model becomes ambiguous. A new developer tasked with creating a marketing email list could easily—and incorrectly—target both groups, because the code itself has lost the context of this vital business distinction \cite{24}.
\end{enumerate}

\section{Formalization: The Contextual Integrity Score (CIS)}

To move beyond the subjectivity of traditional code quality metrics, we define \textbf{Contextual Integrity ($CI$)} not as a qualitative sentiment, but as a computable probability that a given software artifact preserves its original intent.

We propose that the "explainability" of a system is a function of the semantic density between its executable logic (Code) and its declarative intent (Rationale). In AgentBeats, we formalize the \textbf{Contextual Integrity Score ($CIS$)} for any given change set $\Delta$ as a weighted summation of four measurements:

\begin{equation}
CIS(\Delta) = 0.25 \cdot R(\Delta) + 0.25 \cdot A(\Delta) + 0.25 \cdot T(\Delta) + 0.25 \cdot L(\Delta)
\end{equation}

This "Equal Weighting" strategy ensures that no single dimension dominates the score, forcing the system to balance intent, implementation, verification, and logic.

\subsection{Rationale Integrity ($R$): The Semantic Alignment Vector}

Traditional traceability links (e.g., "referencing a Jira ticket") are susceptible to gaming. A developer can link irrelevant tickets to satisfy a compliance check. To solve this, we define Rationale Integrity ($R$) using high-dimensional vector similarity.

Let $\vec{v}_{code}$ be the vector embedding of the generated code diff, and $\vec{v}_{intent}$ be the aggregated vector embedding of the associated requirements (tickets, PRDs, Slack threads). The Rationale Integrity score is the cosine similarity between execution and intent:

\begin{equation}
R(\Delta) = \frac{\vec{v}_{code} \cdot \vec{v}_{intent}}{\|\vec{v}_{code}\| \|\vec{v}_{intent}\|} \cdot \mathbb{I}(Link_{exists})
\end{equation}

Where $\mathbb{I}$ is an indicator function that returns 0 if no formal link exists. This effectively measures: \textit{Does the code mathematically implement the meaning of the requirement?}

\subsection{Architectural Integrity ($A$): The Graph Centrality Constraint}

We model the software architecture as a directed graph $G = (V, E)$, where $V$ represents services and $E$ represents allowed dependencies. Contextual Debt often manifests as "illegal edges"---hidden dependencies that violate architectural constraints (e.g., a frontend service querying a payment database directly).

For a proposed change $\Delta$ introducing a set of new edges $E_{new}$, the Architectural Integrity is calculated as a binary compliance function scaled by the centrality of the violated nodes:

\begin{equation}
A(\Delta) = \prod_{e \in E_{new}} (1 - P(Violation | e))
\end{equation}

This function forces the score to 0 if a single critical architectural boundary is crossed (a "Critical Veto"), ensuring that locally valid code is rejected if it introduces global fragility.

\subsection{Testing Integrity ($T$): Empirical Verification}

Traditional code coverage measures \textit{execution} (did the line run?), not \textit{verification} (did it do the right thing?). In AgentBeats, Testing Integrity ($T$) is derived primarily from the \textbf{ground-truth pass rate} of the code in a secure Docker sandbox, augmented by a "Specificity Bonus" for well-structured test assertions.

\begin{equation}
T(\Delta) = \text{Base}(PassRate) + \text{Bonus}(\text{Specificity})
\end{equation}

This ensures that code that fails to run or fails its own tests cannot achieve a high integrity score, regardless of how "clean" it looks.

\subsection{Logic Integrity ($L$): The Senior Review}

While vectors and sandboxes capture structure and function, they can miss subtle algorithmic flaws or edge cases. Logic Integrity ($L$) is computed via an LLM-based "Senior Code Review" that evaluates the submission for:
\begin{enumerate}
    \item \textbf{Edge Case Handling:} Nulls, empty inputs, boundary values.
    \item \textbf{Algorithmic Complexity:} Efficiency and simplicity.
    \item \textbf{Constraint Adherence:} Compliance with task-specific rules.
\end{enumerate}

\begin{equation}
L(\Delta) = \text{LLM}_{Judge}(\text{Code}, \text{Task})
\end{equation}

\section{The Protocol: Agent-as-a-Judge Architecture}

To enforce the Contextual Integrity Score ($CIS$) defined in Section 3, we propose a governance architecture that inverts the traditional "Black Box" model of AI code generation. We define this architecture as the \textbf{"Glass Box" Protocol}, characterized by the externalization of reasoning and the cryptographic verification of intent.

\subsection{The Orchestrator-Worker Topology}

Traditional AI coding assistants operate as monolithic agents, where the reasoning state is hidden within the model's weights. This opacity renders auditability impossible.

We employ a \textbf{Multi-Agent Orchestrator-Worker (MA-OW)} topology where governance is structural, not supervisory.

\begin{itemize}
    \item \textbf{The Orchestrator ($O$):} Maintains the state of the global context graph $G$. It does not generate code; it generates constraints.
    \item \textbf{The Worker ($W$):} A specialized agent (e.g., Purple Agent / Defender) that generates code $\Delta$ solely to satisfy constraints passed by $O$.
    \item \textbf{The Judge ($J$):} The Green Agent, an adversarial evaluator that runs $\Delta$ in a secure Docker sandbox and computes the $CIS$ function.
\end{itemize}

The interaction is defined by the tuple $(C, \Delta, V)$, where $C$ is the constraint, $\Delta$ is the artifact, and $V$ is the verification vector. The system accepts $\Delta$ if and only if:

\begin{equation}
J(O(C), W(\Delta)) \to \{Accept \mid Reject\}
\end{equation}

\subsection{The Decision Bill of Materials (DBOM)}

In current CI/CD pipelines, the "Software Bill of Materials" (SBOM) tracks \textit{components} (what). This is insufficient for liability protection. AgentBeats generates a \textbf{Decision Bill of Materials (DBOM)} to track \textit{causality} (why).

A DBOM is an immutable, append-only log of the reasoning trace that authorized a code change. For every commit $\Delta_i$, the DBOM records a cryptographic tuple:

\begin{equation}
DBOM_i = \langle H(\Delta_i), \vec{v}_{intent}, \text{Score}_{CIS}, \sigma_{Judge} \rangle
\end{equation}

Where:
\begin{itemize}
    \item $H(\Delta_i)$ is the SHA-256 hash of the full evaluation result (JSON), ensuring no tampering with the score.
    \item $\vec{v}_{intent}$ is the vector embedding of the requirement (the "Why"), allowing mathematical audit of intent alignment.
    \item $\text{Score}_{CIS}$ is the computed Contextual Integrity Score.
    \item $\sigma_{Judge}$ is the digital signature of the Agent-as-a-Judge, certifying the evaluation.
\end{itemize}

This structure ensures that every line of code is mathematically bound to its justification. In the event of a liability claim, the organization does not produce a \textit{log}; it produces a \textbf{proof} of diligence.

\section{Conclusion: Navigating the Decade of Unknowable Code}

This analysis has sought to define and explore the hypothesis that Contextual Debt---the future cost incurred from a lack of discernible human intent, architectural rationale, and domain-specific knowledge---is poised to become the single greatest source of software liability in the coming decade. The evidence suggests that this is not merely a new piece of jargon but a necessary term to describe a distinct and far more dangerous form of liability than traditional technical debt. While technical debt compromises a system's efficiency, Contextual Debt compromises its integrity, safety, and defensibility.

\begin{thebibliography}{99}

\bibitem{1} What is Technical Debt? Definition, History, and Strategy | Praxent, accessed October 26, 2025, \url{https://praxent.com/blog/brief-history-technical-debt}
\bibitem{2} Introduction to the Technical Debt Concept | Agile Alliance, accessed October 26, 2025, \url{https://www.agilealliance.org/wp-content/uploads/2016/05/IntroductiontotheTechnicalDebtConcept-V-02.pdf}
\bibitem{3} medium.com, accessed October 26, 2025, \url{https://medium.com/@yvan.callaou/our-systems-have-no-memory-i-call-it-contextual-debt-d656af68d80b}
\bibitem{4} Why AI-Generated Code Hurts Your Exit – The Bootstrapped Founder, accessed October 26, 2025, \url{https://thebootstrappedfounder.com/why-ai-generated-code-hurts-your-exit/}
\bibitem{5} Tracing the Footsteps of Technical Debt in Microservices: A Preliminary Case Study - Roberto Verdecchia, accessed October 26, 2025, \url{https://robertoverdecchia.github.io/papers/QUALIFIER_2023.pdf}
\bibitem{6} Technical Debt in Microservices: A Mixed-Method Case Study - Roberto Verdecchia, accessed October 26, 2025, \url{https://robertoverdecchia.github.io/papers/QUALIFIER_2024.pdf}
\bibitem{7} The Future of Software Liability: How Software Vendors Can ..., accessed October 26, 2025, \url{https://www.veritas.com/blogs/the-future-of-software-liability-how-software-vendors-can-prepare-for-increasing-responsibility}
\bibitem{8} Bugs in the Software Liability Debate - Just Security, accessed October 26, 2025, \url{https://www.justsecurity.org/87294/bugs-in-the-software-liability-debate/}
\bibitem{9} What Is Technical Debt? | Definition and Examples - ProductPlan, accessed October 26, 2025, \url{https://www.productplan.com/glossary/technical-debt/}
\bibitem{10} If you want to address tech debt, quantify it first - Stack Overflow, accessed October 26, 2025, \url{https://stackoverflow.blog/2023/08/24/if-you-want-to-address-tech-debt-quantify-it-first/}
\bibitem{11} Technical debt - Wikipedia, accessed October 26, 2025, \url{https://en.wikipedia.org/wiki/Technical_debt}
\bibitem{12} What is Technical Debt? | IBM, accessed October 26, 2025, \url{https://www.ibm.com/think/topics/technical-debt}
\bibitem{13} tagged by: technical debt - Martin Fowler, accessed October 26, 2025, \url{https://martinfowler.com/tags/technical%20debt.html}
\bibitem{14} Technical Debt Quadrant - Martin Fowler, accessed October 26, 2025, \url{https://martinfowler.com/bliki/TechnicalDebtQuadrant.html}
\bibitem{15} Technical Debt: Definition + Management | monday.com Blog, accessed October 26, 2025, \url{https://monday.com/blog/rnd/technical-debt/}
\bibitem{16} Technical Debt Explained - Codacy | Blog, accessed October 26, 2025, \url{https://blog.codacy.com/technical-debt}
\bibitem{17} From Technical Debt to Design Integrity | by Alberto Brandolini - Medium, accessed October 26, 2025, \url{https://medium.com/@ziobrando/from-technical-debt-to-design-integrity-48e7056b6776}
\bibitem{18} Technical Debt is over-used : r/PHP - Reddit, accessed October 26, 2025, \url{https://www.reddit.com/r/PHP/comments/1hgxmap/technical_debt_is_overused/}
\bibitem{19} What is Technical Debt? Causes, Types & Definition Guide - Sonar, accessed October 26, 2025, \url{https://www.sonarsource.com/resources/library/technical-debt/}
\bibitem{20} How to Tackle Technical Debt and Maintain High Software Quality - Qt, accessed October 26, 2025, \url{https://www.qt.io/quality-assurance/blog/how-to-tackle-technical-debt}
\bibitem{21} Our Systems Have No Memory. I Call It Contextual Debt | by Yvan Callaou - Medium, accessed October 26, 2025, \url{https://medium.com/@yvan.callaou/our-systems-have-no-memory-i-call-it-contextual-debt-d656af68d80b}
\bibitem{22} Understanding Technical Debt. Technical debt, a concept ..., accessed October 26, 2025, \url{https://jatinmishra27.medium.com/understanding-technical-debt-6fe846fde3a8}
\bibitem{23} Your AI Assistant is a Genius with Amnesia: How to Onboard It | by Krzyś | Generative AI, accessed October 26, 2025, \url{https://generativeai.pub/why-your-ai-assistant-writes-idiotic-code-and-how-to-fix-it-4512b2b5ceb5}
\bibitem{24} Mastering Domain-Driven Design: A Strategic Framework for ..., accessed October 26, 2025, \url{https://medium.com/@Kirtiswagat/the-automation-garage-domain-driven-design-architecture-for-the-financial-industry-c524cf757e14}
\bibitem{25} Examples of Technical Debt's Cybersecurity Impact - DTIC, accessed October 26, 2025, \url{https://apps.dtic.mil/sti/trecms/pdf/AD1144728.pdf}
\bibitem{26} Impact of technical debt: how to identify and reduce it - Paddle, accessed October 26, 2025, \url{https://www.paddle.com/resources/technical-debt}
\bibitem{27} How AI-Generated Code is messing with your Technical Debt - Kodus, accessed October 26, 2025, \url{https://kodus.io/en/ai-generated-code-is-messing-with-your-technical-debt/}
\bibitem{28} The Hidden Costs of Coding With Generative AI - MIT Sloan Management Review, accessed October 26, 2025, \url{https://sloanreview.mit.edu/article/the-hidden-costs-of-coding-with-generative-ai/}
\bibitem{29} Technical Debt and AI: Understanding the Tradeoff and How to Stay Ahead - Qodo, accessed October 26, 2025, \url{https://www.qodo.ai/blog/technical-debt/}
\bibitem{30} Why AI-generated code is creating a technical debt nightmare | Okoone, accessed October 26, 2025, \url{https://www.okoone.com/spark/technology-innovation/why-ai-generated-code-is-creating-a-technical-debt-nightmare/}
\bibitem{31} A Case Study on Technical Debt in a Large-Scale Industrial Microservice Architecture - arXiv, accessed October 26, 2025, \url{https://arxiv.org/html/2506.16214v1}
\bibitem{32} Microservices as Technical Debt: A Critical Perspective | by Dhruv ..., accessed October 26, 2025, \url{https://medium.com/c-sharp-programming/microservices-as-technical-debt-a-critical-perspective-77e8ab9a1a92}
\bibitem{33} (PDF) Architectural Technical Debt in Microservices: A Case Study in a Large Company, accessed October 26, 2025, \url{https://www.researchgate.net/publication/331904375_Architectural_Technical_Debt_in_Microservices_A_Case_Study_in_a_Large_Company}
\bibitem{34} What is Software Scalability and Why is it Important? - CyberlinkASP, accessed October 26, 2025, \url{https://cyberlinkasp.com/what-is-software-scalability-and-why-is-it-important/}
\bibitem{35} Preventing and Reducing Technical Debt at Railsware, accessed October 26, 2025, \url{https://railsware.com/blog/reduce-technical-debt/}
\bibitem{36} Can some explain to me why scalability can be a huge issue in ..., accessed October 26, 2025, \url{https://www.reddit.com/r/learnprogramming/comments/1k2zxz/can_some_explain_to_me_why_scalability_can_be_a/}
\bibitem{37} Impact of Technical Debt in Software Development | by TopDevelopers.co - Medium, accessed October 26, 2025, \url{https://medium.com/@topdevelopers-co/impact-of-technical-debt-in-software-development-af6d0c6c051d}
\bibitem{38} Managing the Consequences of Technical Debt: 5 Stories from the Field, accessed October 26, 2025, \url{https://www.sei.cmu.edu/blog/managing-the-consequences-of-technical-debt-5-stories-from-the-field/}
\bibitem{39} Technical Debt Management: The Road Ahead for Successful Software Delivery - arXiv, accessed October 26, 2025, \url{https://arxiv.org/html/2403.06484v1}
\bibitem{40} Understanding Software Liability: Implications for Businesses, accessed October 26, 2025, \url{https://technologybrokers.com.au/the-struggle-for-software-liability-why-it-matters-and-whats-next-for-businesses/}
\bibitem{41} The struggle for software liability: Inside a 'very, very, very hard problem', accessed October 26, 2025, \url{https://therecord.media/cybersecurity-software-liability-standards-white-house-struggle}
\bibitem{42} Three Questions on Software Liability | Lawfare, accessed October 26, 2025, \url{https://www.lawfaremedia.org/article/three-questions-on-software-liability}
\bibitem{43} ADR process - AWS Prescriptive Guidance - AWS Documentation, accessed October 26, 2025, \url{https://docs.aws.amazon.com/prescriptive-guidance/latest/architectural-decision-records/adr-process.html}
\bibitem{44} Architectural Decision Records, accessed October 26, 2025, \url{https://adr.github.io/}
\bibitem{45} Architecture decision record (ADR) examples for software planning, IT leadership, and template documentation - GitHub, accessed October 26, 2025, \url{https://github.com/joelparkerhenderson/architecture-decision-record}
\bibitem{46} Architecture decision record - Microsoft Azure Well-Architected Framework, accessed October 26, 2025, \url{https://learn.microsoft.com/en-us/azure/well-architected/architect-role/architecture-decision-record}
\bibitem{47} Documenting Architectural Decisions With ADRs A Practical Guide - Growing Scrum Masters, accessed October 26, 2025, \url{https://www.growingscrummasters.com/keywords/architecture-decision-records/}
\bibitem{48} Understanding Domain-Driven Design: A Practical Approach for ..., accessed October 26, 2025, \url{https://sensiolabs.com/blog/2024/understanding-domain-driven-design}
\bibitem{49} Domain-Driven Design Demystified - NDepend Blog, accessed October 26, 2025, \url{https://blog.ndepend.com/domain-driven-design-demystified/}
\bibitem{50} 10 Things to Avoid in Domain-Driven Design (DDD) - DZone, accessed October 26, 2025, \url{https://dzone.com/articles/10-things-to-avoid-in-domain-driven-design}
\bibitem{51} Artificial Intelligence and Technical Debt: Navigating the New Frontier - CAST Software, accessed October 26, 2025, \url{https://www.castsoftware.com/pulse/artificial-intelligence-and-technical-debt-navigating-the-new-frontier}

\end{thebibliography}

\end{document}
